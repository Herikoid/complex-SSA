\documentclass[10pt, ucs, notheorems, handout]{beamer}

\usetheme[numbers,totalnumbers,minimal,nologo]{Statmod}
\usefonttheme[onlymath]{serif}
\setbeamertemplate{navigation symbols}{}
\setbeamercolor{alerted text}{fg=blue}
\usepackage{physics}

\mode<handout> {
    \usepackage{pgfpages}
    %\setbeameroption{show notes}
    %\pgfpagesuselayout{2 on 1}[a4paper, border shrink=5mm]
    \setbeamercolor{note page}{bg=white}
    \setbeamercolor{note title}{bg=gray!10}
    \setbeamercolor{note date}{fg=gray!10}
}

\usepackage[utf8x]{inputenc}
\usepackage[T2A]{fontenc}
\usepackage[english, russian]{babel}
\usepackage{tikz}
\usepackage{ragged2e}
\usepackage{graphicx}
\usepackage{subfigure}


\usepackage{latexsym,amssymb}
\usepackage{amsmath}
\usepackage{amsfonts}
\usepackage{amsthm}

\DeclareMathOperator{\rk}{rk}
\DeclareMathOperator{\med}{med}
\DeclareMathOperator{\diag}{diag}
\DeclareMathOperator*{\argmin}{argmin}
\newcommand{\tX}[1]{\mathsf{#1}}
\newcommand{\iu}{\mathrm{i}\mkern1mu}
\newcommand{\RomanNumeralCaps}[1]
{\MakeUppercase{\romannumeral #1}}

\title[Робастные варианты метода SSA]{%
	Робастные варианты метода SSA}

\author{Сенов Михаил Андреевич}

\institute[СПбГУ]{Санкт-Петербургский государственный университет \\
	%Математико-механический факультет \\
	%Кафедра статистического моделирования \\
	Уровень образования: бакалавриат\\
	Направление 01.03.02 <<Прикладная математика и информатика>>\\
	Основная образовательная программа СВ.5004.2018 <<Прикладная математика и информатика>> \\
	Профессиональная траектория <<Вычислительная стохастика и статистические модели>>\\
	\vspace{0.4cm}
	Научный руководитель: к.ф.-м.н., доц. Голяндина Н.Э. \\
	Рецензент: к.ф.-м.н. Пепелышев А.Н.
	\vspace{0.3cm}
}

\date[Защита]{Санкт-Петербург, 2022}

\begin{document}

\begin{frame}
  \titlepage
  \note{Здравствуйте, я расскажу о своей выпускной квалификационной работе по теме Робастные варинаты метода SSA. Работа выполнена под руководством Голяндиной Н.Э.}
\end{frame}

\begin{frame}{Введение}

$\tX{X} = (x_1, \ldots, x_{N})$ временной ряд длины $N$.\\
\vspace{1em}
\alert{Модель:} $\tX{X} = \tX{S} + \tX{R}$, $\tX{S}$ сигнал, $\tX{R}$ возмущение (шум или выброс).\\
\vspace{1em}
\alert{Задача:} Оценить сигнал $\tilde{\tX{S}} = F(\tX{X})$, $F$ --- используемый метод.\\
\vspace{1em}
\alert{Метод:} SSA (Singular Spectrum Analysis) для вещественных рядов, CSSA (Complex Singular Spectrum Analysis) --- обобщение для комплексных рядов.\\
\textit{\small{Golyandina, Nekrutkin, Zhigljavsky (2001). Analysis of time series structure: SSA and
related techniques.}}\\
\vspace{1em}
\alert{Вопросы:}
\begin{enumerate}
	\item Устойчивые к выбросам (робастные) модификации CSSA?. Робастные модификации SSA (вещ. случай) были предложены ранее (Третьякова, 2020).
	\item Что лучше, с точки зрения величины ошибки $\tilde{\tX{S}} - \tX{S}$,\\
	SSA$(\tX{X}_{\Re}) + \iu \text{SSA}(\tX{X}_{\Im})$ или CSSA$(\tX{X}_{\Re} + \iu \tX{X}_{\Im})$?
\end{enumerate}
\note{Рассмотрим временной ряд длины $N$. В качестве модели возьмём представление временного ряда в виде суммы сигнала и возмущения. В данной работе, в качестве возмущения будет рассмотрен шум(случайная ошибка) и выброс(аномальный элемент). Задачей является оценка сигнала $\tX{S}$ по ряду $\tX{X}$. Данную задачу решает метод SSA в вещественном случае и его комплексное обобщение CSSA в комплексном. В работе рассмотрены две проблемы. Это обобщение рассмотренных Третьяковой робастных модификаций SSA на комплексный случай и вопрос осмысленности комплексного обобщения в принципе. Ведь комплексный временной ряд состоит из вещественной и мнимой частей и разумным является вопрос, а что лучше, с точки зрения ошибки восстановления, применить SSA к вещественной и мнимой частям или применить CSSA ко всему ряду?}
\end{frame}

\begin{frame}{Часть \RomanNumeralCaps{1}. Robust CSSA: Введение}
	Базовый SSA: реагирует на выбросы.
	\begin{center}
		\includegraphics[, height = 3cm]{img/outliers_ssa.PNG}
	\end{center}
	
	Робастный SSA: не реагирует на выбросы.
	\begin{center}
		\includegraphics[, height = 3cm]{img/outliers_l1.PNG}
	\end{center}
	
	\note{Рассмотрим решение первой проблемы. Зачем в принципе нужны робастные модификации SSA? Как показано на слайде, SSA реагирует на выбросы, тогда как робастные модификации нет.}
\end{frame}

%\begin{frame}{Введение: Цель работы}
%	Робастный SSA разработан для вещественных временных рядов.\\
%	\vspace{1em}
%	\alert{Цель:} Обобщить временной ряд на комплексный случай.\\
%	\vspace{1em}
%	Распространённый пример комплексного временного ряда~--- это инженерные задачи, где данные получаются путём преобразования Фурье.
%	$$\tX{X}_N^{(t)} = (x_1^{(t)}, \ldots, x_N^{(t)}), \, t = 1 \ldots M,$$
%	$$\tX{DFT}(\tX{X}_N^{(t)}) = f^{(t)}_k = \sum_{n=1}^{N}x^{(t)}_n e^{-2 \pi i k  n / N}, \, k = 1 \ldots N,$$
%	$$\tX{F}^{(M)}_{k} = (f^{(1)}_k, \ldots, f^{(M)}_k).$$
%\end{frame}


\begin{frame}{Часть \RomanNumeralCaps{1}: Обозначения}
	Рассмотрим временной ряд $\tX{X}=(x_1, \ldots, x_{N}) = \tX{S} + \tX{R}$, $L$ длина окна, $K=N-L+1$.\\
	\vspace{1em}
	Оператор вложения $\mathcal{T}_L:\mathbb{R}^N \rightarrow \mathcal{M}_{\mathcal{H}}: \mathcal{T}_L (\tX{X}) = \mathbf{X} $,\\
	$$\mathbf{X} = \begin{pmatrix}
		x_1 & x_2 & \ldots & x_{K}\\
		x_2 & x_3 & \ldots & x_{K+1}\\
		\vdots & \vdots & & \vdots\\
		x_{L} & x_{L+1} & \ldots & x_{N}
	\end{pmatrix}, K = N - L + 1,$$
	$\mathbf{X}$~--- $L$-траекторная матрица $\tX{X}$.\\ 
	\vspace{1em}
	Ранг сигнала $\rk\mathcal{T}_L (\tX{S}) = r$.\\
	
	\vspace{1em}
	$\mathcal{M}_{r}$ --- пространство матриц размера $L \times K$ ранга не больше $r$.
	
	$\mathcal{M}_{\mathcal{H}}$~--- пространство ганкелевых матриц $L\times K$.
	
	$\mathcal{T}_L (\tX{S}) \in \mathcal{M}_{\mathcal{H}} \cap \mathcal{M}_{r}$.\\
	\vspace{1em}
	$\Pi_{r}:\mathcal{M}\rightarrow \mathcal{M}_r$,
	$\Pi_{\mathcal{H}}:\mathcal{M} \rightarrow \mathcal{M}_{\mathcal{H}}$~--- проекторы на $\mathcal{M}_{r}$ и $\mathcal{M}_{\mathcal{H}}$ по некоторой норме.
	
	\note{Для дальнейшего обсуждения потребуется ввести ряд обозначений. Во-первых введём  матричных пространства: ранга не больше $r$ и пространство ганкелевых матриц. Оператор вложения, переводящий ряд в ганкелеву матрицу, называемую траекторной матрицей ряда. И наконец, два матричных проектора по некоторой норме, на пространство ранга не больше $r$ и на пространство ганкелевых матриц.}
	
\end{frame}

\begin{frame}{Часть \RomanNumeralCaps{1}: Выделение сигнала}
	
	Временной ряд $\tX{X}=(x_1, \ldots, x_{N}) = \tX{S} + \tX{R}$, $\rk\mathcal{T}_L (\tX{S}) = r$.
	\begin{block}{Алгоритм для выделения сигнала}
		\begin{equation*}
			\tilde{\tX{S}} = \mathcal{T}_L^{-1} \Pi_{\mathcal{H}} \Pi_{r} \mathcal{T}_L (\tX{X}).
		\end{equation*}
	\end{block}
	
	
	Нормы для $\Pi_r$ и $\Pi_{\mathcal{H}}$:
	\begin{itemize}
		\item $\mathbb{L}_2$(Фробениус): выделение сигнала базовым SSA
		$$\|\mathbf{X}\|_\mathrm{F} = \sqrt{\sum_{i = 1}\sum_{j = 1}|x_{ij}|^2}.$$
		\item $\mathbb{L}_1$: робастная версия SSA
		$$\|\mathbf{X}\|_1 = \sum_{i = 1}\sum_{j = 1}|x_{ij}|.$$
		\item weighted $\mathbb{L}_2$ : робастная версия SSA
		$$\|\mathbf{W}^{1/2}\odot\mathbf{X}\|_\mathrm{F} = \sqrt{\sum_{i = 1}\sum_{j = 1}w_{ij}|x_{ij}|^2},$$
где выбросам соответствуют маленькие веса.
	\end{itemize}
	\note{Используя введённые обозначения мы можем записать алгоритм выделения сигнала следующим образом. То есть сначала ряд приводится к ганкелевой матрицы, после этого ганкелева матрица проецируется на пространство требуемого ранга, полученная матрица приводится к ганкелевой и из неё получается оценка сигнала. Вопрос в выборе нормы для проекторов. В случае базового SSA используется норма Фробениуса. Рассмотренные модификации заключаются в замене нормы на L1-норму и на взвешенную L2-норму, где выбросам соответствуют маленькие веса}
\end{frame}

%\begin{frame}{Методы: Иллюстрирующий пример}
%	\begin{block}{Пример}
%		$(x_1, \ldots, x_N)$, норма $p(x)$, $\argmin\limits_{c}p((x_1 - c, \ldots, x_N - c)) =\, ?$
%	\end{block}
%	Нормы:
%	\begin{itemize}
%		\item $\mathbb{L}_2$. $\overline{x} = \argmin\limits_{c} \sqrt{\sum_{i = 1}^{N} |x_i - c|^2}$. \\
%		Если $x_j \to \infty$, тогда $\overline{x} \to \infty$. Не робастный.
%		\item $\mathbb{L}_1$. $\med = \argmin\limits_{c} \sum_{i = 1}^{N} |x_i - c|$. \\
%		Если $x_j \to \infty$, тогда $\med \not\to \infty$. Робастный.
%		\item weighted $\mathbb{L}_2$. $\overline{x}_w = \argmin\limits_{c} \sqrt{\sum_{i = 1}^{N} w_i|x_i - c|^2}$.\\
%		Если $x_j \rightarrow \infty$, тогда возьмём $w_i = 0$ и $\overline{x}_w \not\to \infty$. Робастный.
%	\end{itemize}
%\end{frame}

\begin{frame}{Часть \RomanNumeralCaps{1}: Результаты}
	Алгоритмы робастных версий CSSA были построены и реализованы на языке R.

	\vspace{1em}
Алгоритмы основаны на решении следующих задач.

	Траекторная матрица $\mathbf{Y} \in \mathbb{C}^{L\times K}$ на вход.
    Разложение $\mathbf{M}\mathbf{V}^{\mathrm{H}}$ на выходе,
	где $\mathbf{M} \in \mathbb{C}^{L\times r}, \mathbf{V} \in \mathbb{C}^{K\times r}, \mathbf{W} \in \mathbb{R}^{L\times K}$.\\
	\vspace{1em}
	\begin{itemize}
		\item $\mathbb{L}_2$ CSSA:
		$$ \|\mathbf{Y}-\mathbf{M}\mathbf{V}^{\mathrm{H}}\|_\mathrm{F} \longrightarrow \min_{\mathbf{M},\mathbf{V}},$$
		имеет решение в замкнутой форме.
		\item $\mathbb{L}_1$ CSSA:
		$$\|\mathbf{Y}-\mathbf{M}\mathbf{V}^{\mathrm{H}}\|_1 \longrightarrow \min_{\mathbf{M},\mathbf{V}}.$$
		решение вычисляется итеративно.
		\item weighted $\mathbb{L}_2$ CSSA:
		$$\|\mathbf{W}^{1/2}\odot(\mathbf{Y}-\mathbf{M}\mathbf{V}^{\mathrm{H}})\|_\mathrm{F} \longrightarrow \min_{\mathbf{M},\mathbf{V}},$$
		решение вычисляется итеративно, $\mathbf{W}$ обновляется на каждой итерации согласно величине остатков разложения.
	\end{itemize}
	
	\note{Ключевым результатом первой части является то, что робастные модификации CSSA были построены и реализованы на языке R. Эти алгоритмы основаны на решении задач минимизации по следующим нормам. L2-норме в случае классического CSSA, L1-норме и взвешенной L2-норме в случае реализованных модификаций. Стоит отметить, что задача в L2-норме имеет решение в замкнутой форме, тогда как задачи для модификаций решаются итеративно.}
\end{frame}

\begin{frame}{Часть \RomanNumeralCaps{2}. Ошибки в SSA и CSSA: Введение}
	
	$\tX{X} = (x_1, \ldots, x_{N})$ временной ряд длины $N$.\\
	\vspace{1em}
	\alert{Модель:} $\tX{X} = \tX{S} + \tX{R}$, $\tX{S}$ сигнал, $\tX{R}$ возмущение (шум или выброс).\\
	\vspace{1em}
	\alert{Задача:} Оценить сигнал $\tilde{\tX{S}} = F(\tX{X})$, $F$ --- используемый метод.\\
	\vspace{1em}
	\alert{Метод:} SSA (Singular Spectrum Analysis) для вещественных рядов, CSSA (Complex Singular Spectrum Analysis) --- обобщение для комплексных рядов.\\
	\textit{\small{Golyandina, Nekrutkin, Zhigljavsky (2001). Analysis of time series structure: SSA and
			related techniques.}}\\
	\vspace{1em}
	\alert{Вопрос:} Что лучше, с точки зрения величины ошибки $\tilde{\tX{S}} - \tX{S}$,\\
	SSA$(\tX{X}_{\Re}) + \iu \text{SSA}(\tX{X}_{\Im})$ или CSSA$(\tX{X}_{\Re} + \iu \tX{X}_{\Im})$?\\
	\vspace{1em}
	Рассматриваем теорию возмущений (Kato, 1966). Будем вычислять первый порядок ошибки (Nekrutkin, 2008), в предположении, что первый порядок достаточно точно описывает полную ошибку.
	
	\note{Перейдём ко второй части работы. Всё так же решается задача оценки сигнала. Только теперь мы рассматриваем вопрос осмысленности комплексных обобщений, с точки зрения ошибки восстановления. Для вычисления ошибки восстановления, мы рассмотрим теорию возмущений и, в силу объективной сложности задачи, будем вычислять не полную ошибку, а её первый порядок, в предположении, что первый порядок достаточно точно описывает полную ошибку.}
\end{frame}

\begin{frame}{Часть \RomanNumeralCaps{2}. Ошибки в SSA и CSSA: Структура}
	Структура Части 2:

\vspace{1em}
	\begin{enumerate}
		\item Первый порядок ошибки для общего случая, теоретическое сравнение SSA и CSSA.

\vspace{1em}
		\item Пример: сигнал --- гармоники, возмущение --- шум. Особый случай комплексной экспоненты.

\vspace{1em}
		\item Пример: сигнал --- константный сигнал, возмущение --- выброс. Явный вид ошибки.

\vspace{1em}
		\item Численное сравнение первого порядка ошибки и полной ошибки.
	\end{enumerate}
	\note{Приведём структуру доклада по второй части. В первую очередь мы рассмотрим теоретическое сравнение CSSA и SSA. После остановимся подробнее на двух примерах, зашумлённых гармоник и константного сигнала с выбросом и, наконец, численно сравним первый порядок ошибки с полной ошибкой}
\end{frame}

\begin{frame}{Часть \RomanNumeralCaps{2}: Обозначения}
Временной ряд $\tX{X}=(x_1, \ldots, x_{N})$, $L$ --- длина окна, $r$ --- ранг оцениваемого сигнала (ранга траекторной матрицы сигнала).
    \begin{block}{Алгоритм SSA (CSSA) выделения сигнала}
\begin{equation*}
	\tilde{\tX{S}} = \mathcal{T}^{-1}_{L} \Pi_{\mathcal{H}} \Pi_{r} \mathcal{T}_L (\tX{X}).
\end{equation*}
\end{block}

\alert{Модель:} $\tX{X} = \tX{S}(\delta)$, где $\tX{S}(\delta) = \tX{S} + \delta \tX{R}$ длины $N$\\ $\mathbf{H} = \mathcal{T}_L(\tX{S})$.\\
\vspace{1em}
$\tilde{\tX{S}} = \mathcal{T}_L^{-1} \Pi_{\mathcal{H}} (\mathbf{H} + \delta\mathbf{H}^{(1)} + \delta^2\mathbf{H}^{(2)})$ из (Nekrutkin, 2008).\\
\vspace{1em}
$\tX{F} = \tilde{\tX{S}} - \tX{S} = \mathcal{T}_L^{-1} \Pi_{\mathcal{H}} (\delta\mathbf{H}^{(1)} + \delta^2\mathbf{H}^{(2)})$ ошибка восстановления.\\
\vspace{1em}
Рассматриваем $\delta = 1$,\\
\fbox{$\tX{F}^{(1)} = \mathcal{T}_L^{-1} \Pi_{\mathcal{H}}(\mathbf{H}^{(1)})$ первый порядок ошибки восстановления.}
\note{Напомним как выглядит оценка сигнала алгоритмом SSA(CSSA). Рассмотрим модель $\tX{S}(\delta)$. Из работы Некруткина известно выражение оценки сигнала через константу по $\delta$, линейную компоненту и квадратичную по $\delta$. Отсюда естественным образом можно выразить ошибку восстановления через линейную и квадратичную компоненты по $\delta$. И взяв $\delta = 1$ обозначим за первый порядок ошибки диагонально усреднённую линейную по $\delta$ компоненту.}
\end{frame}

\begin{frame}{Часть \RomanNumeralCaps{2}: Формула для $\mathbf{H}^{(1)}$}

Итак, рассматриваем $\delta = 1$,
$\tX{X} = \tX{S}(1) = \tX{S} + \tX{R}$,

оценку сигнала 
$\tilde{\tX{S}} = \mathcal{T}_L^{-1} \Pi_{\mathcal{H}} (\mathbf{H} + \mathbf{H}^{(1)} + \mathbf{H}^{(2)}))$.

Хотим найти $\mathbf{H}^{(1)}$, поскольку ошибка $\tX{F} \approx \tX{F}^{(1)} = \mathcal{T}_L^{-1} \Pi_{\mathcal{H}}(\mathbf{H}^{(1)})$.\\

\begin{block}{Утверждение}
	Пусть $\tX{R}$ достаточно мало. Тогда
	\begin{equation*} \label{eq:main}
		\mathbf{H}^{(1)} = \mathbf{P}^{\perp}_0 \mathbf{E} \mathbf{Q}_0 + \mathbf{P}_0 \mathbf{E},
	\end{equation*}
	где $\mathbf{P}_0$~--- проектор на пространство столбцов $\mathbf{H}$, \\$\mathbf{Q}_0$~--- проектор на пространство строк $\mathbf{H}$,\\ $\mathbf{P}^{\perp}_0 = \mathbf{I} - \mathbf{P}_0$, $\mathbf{I}$~--- единичная матрица,\\
	$\mathbf{E} = \mathcal{T}_L(\tX{R})$.
\end{block}
Получено на основе результатов из (Константинов,~2018) и (Некруткин,~2010).

\note{Во введённых выше обозначениях и в предположении, что первый порядок хорошо приближает ошибку восстановления, нас интересует первый порядок ошибки, который является диагонально усреднённой матрицей $\mathbf{H}^{(1)}$. Соответственно, задачей является нахождение данной матрицы. Основываясь на результатах Некруткина и Константинова, в работе была получена следующая формула, выражающая $\mathbf{H}^{(1)}$ один через проекторы на пространство строк и столбцов траекторной матрицы сигнала, а также через траекторную матрицу возмущения.}
\end{frame}

\begin{frame}{Часть \RomanNumeralCaps{2}: Теорема}
	Первые порядки ошибки восстановления:
	\begin{itemize}
	\item CSSA-$\tX{F}^{(1)}$: сигнал $\tX{S}$, возмущение $\tX{R}$, метод CSSA,
	\item SSA-$\tX{F}^{(1)}_{\Re}$: сигнал $\Re(\tX{S})$, возмущение $\Re(\tX{R})$, метод SSA,	
	\item SSA-$\tX{F}^{(1)}_{\Im}$: сигнал $\Im(\tX{S})$, возмущение $\Im(\tX{R})$, метод SSA.
	\end{itemize}

    \begin{block}{Теорема \label{th:sum}}
        Пусть пространства столбцов траекторных матриц рядов $\tX{S}$, $\Re(\tX{S})$ и $\Im(\tX{S})$ совпадают и то же самое верно для пространств строк.
    Тогда при любом достаточно малом возмущении $\tX{R}$
    $$\text{CSSA-}\tX{F}^{(1)} = \text{SSA-}\tX{F}^{(1)}_{\Re} + \iu\text{SSA-}\tX{F}^{(1)}_{\Im}.$$
    \end{block}
    Получается из линейности вхождения $\mathbf{E}$ в формулу
    $$\mathbf{H}^{(1)} = \mathbf{P}^{\perp}_0 \mathbf{E} \mathbf{Q}_0 + \mathbf{P}_0 \mathbf{E}.$$
    \note{Используя линейность вхождения траекторной матрицы возмущения в полученную формулу, была сформулированная следующая теорема. Если пространства траекторных матриц совпадают, то при достаточно малом возмущении первый порядок ошибки для CSSA выражается через сумму первых порядков для SSA. То есть для таких сигналов неважно, какой из подходов использовать.}
\end{frame}

%\begin{frame}{Ошибка восстановления: Случайное возмущение}
%$\text{CSSA-}\tX{F}^{(1)} = (\text{CSSA-}f^{(1)}_1, \ldots, \text{CSSA-}f^{(1)}_N)$,
%
%$\text{SSA-}\tX{F}^{(1)}_{\Re} = (\text{SSA-}f^{(1)}_{\Re, 1}, \ldots, \text{SSA-}f^{(1)}_{\Re, N})$,
%
%$\text{SSA-}\tX{F}^{(1)}_{\Im} = (\text{SSA-}f^{(1)}_{\Im, 1}, \ldots, \text{SSA-}f^{(1)}_{\Im, N})$\\
%\vspace{1em}
%
%Пусть возмущение $\tX{R}$~--- шум, т.е. комплексный случайный вектор с нулевым матожиданием.\\
%\vspace{1em}
%\begin{block}{Применение теоремы}
%Пусть выполнены условия теоремы.
%Тогда
%\begin{equation*} \label{eq:dispsum}
%\mathbb{D}(\text{CSSA-}f^{(1)}_l) = \mathbb{D}(\text{SSA-}f^{(1)}_{\Re, l}) + \mathbb{D}(\text{SSA-}f^{(1)}_{\Im, l}).	
%\end{equation*}
%\end{block}
%
%\alert{Известно:} Пусть $\zeta = \xi + \iu\eta$. Тогда $\mathbb{D}(\zeta) = \mathbb{D}(\xi) + \mathbb{D}(\eta)$.
%
%Такое свойство комплексных случайных величин является объяснением, почему в теореме не требуется независимость вещественной и мнимой частей шума.
%\end{frame}

\begin{frame}{Часть \RomanNumeralCaps{2}: Пример, две зашумлённые синусоиды}
\alert{Сигнал:}
\begin{equation*}
\label{eq:general_ts}
s_l = A\cos(2 \pi\omega l + \phi_1) + \iu B\cos(2 \pi\omega l + \phi_2),
\end{equation*}
где $0<\omega\le 0.5$ и $0\le\phi_i < 2\pi$.

\alert{Особый случай:} При $|\phi_2-\phi_1| = \pi/2$ и $A=B$, комплексная экспонента,
$$s_l = Ae^{\pm \iu(2 \pi\omega l + \phi_1)}.$$
\alert{Возмущение:} Случайный стационарный процесс с нулевым матожиданием и достаточно малой дисперсией.\\
\vspace{1em}
\alert{Обозначения:}

$\text{CSSA-}\tX{F}^{(1)} = (\text{CSSA-}f^{(1)}_1, \ldots, \text{CSSA-}f^{(1)}_N)$,

$\text{SSA-}\tX{F}^{(1)}_{\Re} = (\text{SSA-}f^{(1)}_{\Re, 1}, \ldots, \text{SSA-}f^{(1)}_{\Re, N})$,

$\text{SSA-}\tX{F}^{(1)}_{\Im} = (\text{SSA-}f^{(1)}_{\Im, 1}, \ldots, \text{SSA-}f^{(1)}_{\Im, N})$.

\note{Перейдём к рассмотрению конкретных примеров сигнала, начнём с зашумлённых гармоник. Сигнал две синусоиды, возмущение это шум. Стоит отметить особый случай, при совпадении модулей синусоид и сдвиге в пи пополам, сигнал вырождается в комплексную экспоненту.}
\end{frame}

\begin{frame}{Часть \RomanNumeralCaps{2}: Пример, две зашумлённые синусоиды. MSE}
\begin{block}{Следствие}
    Для сигнала $s_l = A\cos(2 \pi\omega l + \phi_1) + \iu B\cos(2 \pi\omega l + \phi_2)$, не являющегося комплексной экспонентой, с достаточно малым возмущением $\tX{R}$ выполняется
    $$\mathbb{D}(\text{CSSA-}f^{(1)}_l) = \mathbb{D}(\text{SSA-}f^{(1)}_{\Re, l}) + \mathbb{D}(\text{SSA-}f^{(1)}_{\Im, l}).$$
\end{block}

Показано, используя (Степанов,~Голяндина,~2005).

\begin{block}{Предположение}
    Для сигнала $s_l = Ae^{\pm \iu(2 \pi\omega l + \phi_1)}$, с достаточно малым возмущением $\tX{R}$ выполняется
    $$\mathbb{D}(\text{CSSA-}f^{(1)}_l) \stackrel{?}{=} \frac{1}{2}[\mathbb{D}(\text{SSA-}f^{(1)}_{\Re, l}) + \mathbb{D}(\text{SSA-}f^{(1)}_{\Im, l})].$$
\end{block}
Показано эмпирически.

\note{В случае зашумлённой гармоники, не являющейся комплексной экспонентной, дисперсия CSSA равна сумме дисперсий SSA. Это верно из-за того, что выполняется теорема и траекторные пространства совпадают, что было показано в работе Голяндиной и Степанова. В случае же комплексной экспоненты нами было выдвинуто предположение, что дисперсия CSSA равна полусумме дисперсий SSA. Данное предположение было проверено рядом численных экспериментов. Таким образом, применение CSSA для зашумлённых гармоник имеет преимущество только в случае комплексной экспоненты.}

\end{frame}

\begin{frame}{Часть \RomanNumeralCaps{2}: Пример, константный сигнал с выбросом}
\alert{Сигнал:}
$$s_l = c_1 + \iu c_2.$$
\alert{Возмущение:} $\tX{R} = (0, \ldots, a_1 + \iu a_2, \ldots, 0)$~--- достаточно малый выброс на позиции $k$.\\
\vspace{1em}
Тракторные пространства сигнала совпадают, возмущение достаточно малое, справедлива теорема $$\text{CSSA-}\tX{F}^{(1)} = \text{SSA-}\tX{F}^{(1)}_{\Re} + \iu\text{SSA-}\tX{F}^{(1)}_{\Im}.$$

\vspace{1em}
\underline{Достаточно уметь вычислять $\text{SSA-}\tX{F}^{(1)}_{\Re} = \mathcal{T}_L^{-1} \Pi_{\mathcal{H}}(\mathbf{H}^{(1)}_{\Re})$.}

\vspace{1em}
Для $\mathbf{H}^{(1)}_{\Re}$ известна формула (Nekrutkin, 2008)

$$\mathbf{H}^{(1)}_{\Re} = -U^{\mathrm{T}} \mathbf{E} V U V^{\mathrm{T}} + U U^{\mathrm{T}} \mathbf{E} + \mathbf{E} V V^{\mathrm{T}},$$
где $U = \{1/\sqrt{L}\}^{L}_{i = 1},\, V = \{1/\sqrt{K}\}^{K}_{i = 1}$, $K = N - L + 1$.

\note{Рассмотрим случай возмущения выбросом, для простоты рассматриваем константный сигнал. Данный сигнал удовлетворяется теореме и, поэтому, для вычисления первого порядка ошибки для CSSA, нам достаточно знать первый порядок ошибки для SSA. Для этого случая известна формула для выражения $\mathbf{H}^{(1)}$, полученная в работе Некруткина.}
\end{frame}

\begin{frame}{Часть \RomanNumeralCaps{2}: Пример, константный сигнал с выбросом, явный вид SSA-$f^{(1)}_{\Re,l}$}
$\text{SSA-}f^{(1)}_{\Re, l} = \big( \mathcal{T}_L^{-1} \Pi_{\mathcal{H}}(\mathbf{H}^{(1)}_{\Re})\big)_{l}$\\
\vspace{1em}
    Приведем результат для случая $k \leq \min(L/2, K - L)$ и $L < K$, где $K=N-L+1$:
$$\text{SSA-}f^{(1)}_{\Re, l} = \frac{a}{{LK}}
\begin{cases}
	(L + K - k), & \text{$1 \leq l \leq k$}\\
	\frac{1}{l}(L + K - l)k, & \text{$k < l \leq L$}\\
	\frac{1}{L}K(L + k - l), &\text{$L < l < L + k$}\\
	0, &\text{$L + k \leq l \leq K$}\\
	\frac{1}{N - l + 1}(K - l)(L - k), &\text{$K < l < K + k$}\\
	-k, &\text{$K + k \leq l \leq N $}
\end{cases}.$$

\alert{Замечание:} При фиксированном $L$ первый порядок ошибки не стремится к $0$ с ростом $N$.

\note{Подставив в эту формулу конкретные компоненты, и диагонально усреднив полученную матрицу, мы получили аналитический вид для первого порядка ошибки. На слайде приведена формула для выброса на определённой позиции, однако в общем случае она аналогична. Из формулы можно увидеть, что при фиксированном $L$ первый порядок ошибки не стремится к $0$ с ростом $N$. В дальнейшем мы увидим, что это следствие того, что в таком случае первый порядок плохо приближает полную ошибку, поскольку она стремится к $0$.}
\end{frame}

%\begin{frame}{Выброс: График первого порядка для $k < L$}
%Рассмотрим
%$$s_l = 1 + \iu,$$
%$\tX{R}$~--- выброс $10 + \iu 10$ на позиции $k = 1, 5, 10$, $N = 50$, $L = 20$.
%
%% \begin{figure}[H]
%%     \begin{center}
%%         \includegraphics[, height = 2.2cm]{const_outl_graph_1.PNG}
%%         \caption{График $\Re(\tX{X})$.}
%%     \end{center}
%% \end{figure}
%\begin{figure}[H]
%     \begin{center}
%        \includegraphics[, height = 5cm]{img/const_outl_err_1.PNG}
%        \caption{График $\Re(\text{CSSA-}\tX{F}^{(1)})$.}
%    \end{center}
%\end{figure}
%\end{frame}

%\begin{frame}{Выброс: График первого порядка для $L < k < K$}
%Рассмотрим
%$$s_l = 1 + \iu,$$
%$\tX{R}$~--- выброс $10 + \iu 10$ на позиции $k = 26$, $N = 51$, $L = 10, 20, 25$.
%
%% \begin{figure}[H]
%%     \begin{center}
%%         \includegraphics[, height = 2.2cm]{const_outl_graph_2.PNG}
%%         \caption{График $\Re(\tX{X})$.}
%%     \end{center}
%% \end{figure}
%\begin{figure}[H]
%     \begin{center}
%        \includegraphics[, height = 5cm]{img/const_outl_err_2.PNG}
%        \caption{График $\Re(\text{CSSA-}\tX{F}^{(1)})$.}
%    \end{center}
%\end{figure}
%\end{frame}

\begin{frame}{Часть \RomanNumeralCaps{2}: Сравнение первого порядка и полной ошибок}
\alert{Зашумлённые синусоиды:} Численно было показано, первый порядок адекватно описывает полную ошибку.\\
\vspace{1em}
\alert{Константный сигнал с выбросом:} Численно было показано, первый порядок адекватно оценивает полную ошибку при $L = \alpha N$ для больших $N$. При фиксированном $L$ это не так.\\
\vspace{1em}
\alert{Пример с выбросом:}
$$s_l = 1 + \iu,$$
$\tX{R}$~--- выброс $10 + \iu 10$ на позиции $k = L - 1$.

\begin{table}[H]
	\begin{center}
		\caption{Максимальное различие первого порядка и полной ошибок.}
		\label{tab:const_outl}
		\begin{tabular}{|c|c|c|c|c|}
			\hline
			$N$	& 50 & 100 & 400 & 1600 \\
			\hline
			$L = N / 2$ & 0.1313  & 0.0419  & 0.0033 & 0.0002 \\
			\hline
			$L = 20$ & 0.3074  & 0.1965  & 0.5655 & 0.6720 \\
			\hline
		\end{tabular}
	\end{center}
\end{table}

\note{Приведённые рассуждения имели смысл в предположении, что первый порядок ошибки достаточно точно приближает полную ошибку, но так ли это? Численные эксперименты показывают, что для случая зашумлённых гармоник это так. Тогда как для случая константного сигнала с выбросом это выполняется в случае пропорциональности длины окна длине ряда и не выполняется при малой длине окна, что представлено в таблице для конкретного примера.}
\end{frame}


\begin{frame}{Основные результаты}
    \begin{itemize}
    	\item \alert{Реализации на R:} Робастные модификации были обобщены на комплексный случай.
        \item \alert{Теория:} При совпадении траекторных пространств комплексного ряда первые порядки ошибок для CSSA и SSA совпадают.
        \item \alert{Пример:} \begin{itemize}
        	\item $s_l = A\cos(2 \pi\omega l + \phi_1) + \iu B\cos(2 \pi\omega l + \phi_2)$, сдвиг не $\pi / 2$
        	$$\mathbb{D}(\text{CSSA-}f^{(1)}_l) = \mathbb{D}(\text{SSA-}f^{(1)}_{\Re, l}) + \mathbb{D}(\text{SSA-}f^{(1)}_{\Im, l}).$$
        	\item $s_l = Ae^{\pm \iu(2 \pi\omega l + \phi_1)}$
        	$$\mathbb{D}(\text{CSSA-}f^{(1)}_l) \stackrel{?}{=} \frac{1}{2}[\mathbb{D}(\text{SSA-}f^{(1)}_{\Re, l}) + \mathbb{D}(\text{SSA-}f^{(1)}_{\Im, l})].$$
        \end{itemize}
        \item \alert{Пример:} Для $s_l = c_1 + \iu c_2$ с выбросом была получена аналитическая формула первого порядка. При $L = \alpha N$ ошибка стремится к $0$.
        \item \alert{Численные эксперименты:} Рассмотрен вопрос приближения полной ошибки первым порядком.
    \end{itemize}

	\note{В заключение приведём основные результаты. Робастные модификации CSSA были построены и реализованы на R. Был получен теоретический результат о совпадении первых порядков ошибок для SSA и CSSA, в случае совпадения траекторных пространств. Для сравнения дисперсий CSSA и SSA для зашумлённых гармоник, не являющихся комплексной экспонентой был получен теоретический результат и эмпирический результат для комплексной экспоненты. Был получен явный вид первого порядка ошибки для комплексного сигнала с выбросом. И наконец, был рассмотрен вопрос приближения полной ошибки первым порядком. На этом всё, спасибо за внимание.}
\end{frame}

%\begin{frame}{Список литературы}
%\begin{thebibliography}{10}
%{\small
%	\bibitem{Golyandina.etal2013}
%	N.~Golyandina, A.~Korobeynikov, A.~Shlemov, and K.~Usevich (2015)
%	\newblock Multivariate and {2D} extensions of singular spectrum analysis with
%	the {R}ssa package.
%	
%
%	\bibitem{Golyandina.etal2001}
%	N.~Golyandina, V.~Nekrutkin, and A.~Zhigljavsky (2001)
%	\newblock {\em Analysis of Time Series Structure: {SSA} and Related
%		Techniques}.
%
%	
%	\bibitem{Golyandina.Stepanov2005}
%	Д.~Степанов, Н.~Голяндина (2005)
%	\newblock Варианты метода "Гусеница"{-SSA} для прогноза многомерных временных рядов.
%
%	\bibitem{Nekrutkin}
%	V.~Nekrutkin (2010)
%	\newblock Perturbation expansions of signal subspaces for long signals.
%
%	
%	\bibitem{NekrutkinPerp}
%	V.~Nekrutkin (2008)
%	\newblock Perturbations in SSA.
%
%	
%	\bibitem{Konstantinov}
%	А.~Константинов (2018)
%	\newblock Некоторые задачи анализа временных рядов.
%}	
%\end{thebibliography}
%\end{frame}

\end{document}
