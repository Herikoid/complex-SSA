\documentclass[12pt,a4paper]{article}
\usepackage[english,russian]{babel} 

\linespread{1.5}

\usepackage[a4paper,
mag=1000, includefoot,
left=3cm, right=1.5cm, top=2cm, bottom=2cm, headsep=1cm, footskip=1cm]{geometry}

\usepackage{latexsym,amssymb, amsthm}
\usepackage{subcaption}
\usepackage{physics}
\usepackage{amsmath}
\usepackage{graphicx}
\usepackage{float}
\input{letters_series_mathbb}
\input{newcommands}
\newtheorem{proposition}{Предложение}%[section]
\newtheorem{corollary}{Следствие}%[section]
\newtheorem{theorem}{Теорема}%[section]
\newtheorem{remark}{Замечание}%[section]
\newtheorem{lemma}{Лемма}%[section]
\newtheorem{definition}{Определение}%[section]
\newtheorem{algorithm}{Алгоритм}%[section]
\newtheorem*{prop*}{Предположение}

\newcommand{\sfS}{\sf S}
\newcommand{\bfPi}{\mathbf{\Pi}}
\newcommand{\bfZero}{\mathbf{0}}
\newcommand{\argmin}{\mathop{\rm argmin}}
\newcommand{\ord}{\mathop{\rm ord}}
\newcommand{\fdim}{\mathop{\rm fdim}}
%\newcommand{\eqref}[1]{(\ref{#1})}


\title{Ошибки оценивания сигнала с помощью комплексного анализа сингулярного спектра
%\title{Ошибки оценивания сигнала с помощью комплексного анализа сингулярного спектра для зашумленной суммы комплексных экспонент.
}

%\author{-}
\date{2022}

\begin{document}

\maketitle

\section{Введение}

%Анализ cингулярного cпектра (singular spectrum analysis, SSA)  \cite{Golyandina.etal2001}
% --- мощный метод анализа временных
%рядов, не требующий предварительного задания параметрической модели ряда. Метод имеет естественное расширение на случай комплексных временных рядов, называемое Complex SSA (CSSA).
%Есть класс сигналов, а именно временные ряды, управляемые линейными рекуррентными соотношениями, который позволяет получать для него теоретические результаты относительно свойств метода SSA.

Временным рядом называют набор наблюдений сделанных, как правило, в равноотстоящие промежутки времени. В данной работе будем рассматривать комплексный временной ряд, то есть ряд с наблюдениями в виде комплексных чисел.

Такие ряды регулярно возникают в инженерных задачах. Из-за неточности приборов, человеческого фактора и других причин, к интересующей исследователей информации нередко добавляются ошибки. Полезную часть ряда называют сигналом, ошибки возмущением. Из этого возникает задача выделения сигнала из временного ряда. Данная задача является актуальной и была рассмотрена во многих статьях, см., например,~\cite{8683056},~\cite{MOOERS19731129}. Одним из методов, применяющихся для решения задачи выделения сигнала, является анализ сингулярного спектра (SSA) и его комплексное обобщение, CSSA, см., например,~\cite{7337988},~\cite{Chen}.

Рассуждая о конкретных методах, немаловажным является вопрос вычисления ошибки оценки сигнала (ошибки восстановления), поскольку это даёт возможность оценить эффективность метода. Для CSSA было произведено численное исследование ошибки восстановления (\cite{Golyandina.etal2013}). Однако, насколько мне известно, теоретических результатов по оценке ошибки восстановления не было получено. Исходя из этого, целью данной работы было теоретическое изучение ошибки восстановления метода CSSA.

Рассмотрим комплексный временной ряд вида $\tX =\tS + \tR$. Будем использовать метод CSSA для получение оценки сигнала $\wtilde\tS$. Помимо применения CSSA ко всему ряду, будем также применять метод SSA отдельно к вещественной и мнимой частям ряда $\tX$.

Для анализа ошибки восстановления в работе используется теория возмущений \cite{Kato}, которая была применена для случая выделения сигнала методом SSA в ряде работ, см., например, \cite{Nekrutkin}.

Не смотря на то, что теория возмущения Като дает вид полной ошибки, её исследование представляется сложной задачей. Исходя из этого, в работе рассмотрен только первый порядок ошибки в разложении ошибки восстановления по величине возмущения.

Помимо этого, проведем численное сравнение первого порядка и полной ошибок для выявления случаев, когда анализ первого порядка ошибки плохо описывает полную ошибку, в связи с чем анализ первого порядка не представляет интереса.

Получение явного вида ошибки --- это довольно трудоемкая задача даже для первого порядка. В работе удалось его получить для случая константного сигнала с возмущением в виде выброса. В общем случае результаты касаются сравнения MSE ошибок оценки сигнала методом CSSA и суммарного MSE при применении SSA отдельно к мнимой и вещественной частям.

Структура работы следующая. В секции \ref{sec:basessa} представлены необходимы сведения о методе CSSA и ранге ряда. В секции \ref{sec:cssa_theor} приведены полученные теоретические результаты для ошибок оценки сигнала методом CSSA, а также проведено теоретическое сравнение ошибок оценки сигнала методом CSSA и SSA, применённого отдельно к вещественной и мнимой частям. Секция \ref{sec:cssa_comp} содержит сравнение реализаций CSSA и SSA по трудоёмкости. И, наконец, секция \ref{sec:results} содержит численное сравнение первого порядка и полной ошибок восстановления.

%В комплексных временных рядах одним из часто встречающихся случаев является сигнал, состоящий из суммы комплексных экспонент.

%\section{Элементы теории возмущений}
%\label{sec:pert}
%TODO: нужно это?

\section{Алгоритм CSSA}
\label{sec:basessa}

В данном разделе приведём алгоритм CSSA (\cite{Golyandina.etal2001}).

\begin{algorithm}
\label{alg:ssa}
~\\
\textbf{Вход:} Комплексный временной ряд $\tX = (x_1, \dots, x_N)$, \emph{длина окна} $L$,
    ранг сигнала $r$.\\
\textbf{Выход:} Оценка сигнала $\wtilde\tS$.\\
\textbf{Алгоритм:}
\begin{enumerate}
    \item \textbf{Вложение.}
        \label{item:embedding}
        Построим $\bfX \in \spaceR^{L \times K}$, \emph{$L$-траекторную матрицу} ряда $\tX$:
        $$\bfX = \calT_L \tX = [X_1 : \dots X_K],$$
        где $K = N - L + 1$,
        а $X_i$ --- \emph{векторы $L$-вложения:}
        $X_i = (x_i, \dots, x_{i+L-1})^\mathrm{T} \in \spaceR^L$.
    \item \textbf{Разложение.}
        \label{item:decomposition}
        Построим SVD-разложение матрицы $\bfX$:
        $\bfX =
        \sum_{k=1}^{\rank \bfX} \sqrt{\lambda_k} U_k V_k^\mathrm{H} = \sum_{k=1}^{\rank \bfX} \widehat{\bfX}_k$,
        где $U_k$, $V_k$ --- правые и левые сингулярные векторы матрицы $\bfX$ соответственно,
        $\sqrt{\lambda_k}$ --- сингулярные числа.
    \item \textbf{Группировка.} Сгруппируем матрицы компонент сигнала $\widehat{\bfS}$:
        $\widehat{\bfS} = \sum_{k=1}^r \widehat{\bfX}_k$.
    \item \textbf{Диагональное усреднение.}
        \label{item:reconstruction}
        Применим процедуру диагонального усреднения (проекции в норме Фробениуса
        на линейное пространство ганкелевых матриц):
        $\widetilde{\bfS} = \calH \widehat{\bfS}$,
        затем сопоставим полученным Ганкелевым матрицам ряды длины~$N$:
        $\widetilde{\tS} = \calT_L^{-1} \widetilde{\bfS}$.
\end{enumerate}
\end{algorithm}

$L$-Рангом временного ряда называется ранг его траекторной матрицы. Для дальнейших рассуждений потребуется знание рангов конкретных рядов.
Из \cite{Golyandina.Stepanov2005} известно, что ранг комплексного сигнала, вещественная и мнимая чать которого являются синусоидами с одинаковой частотой $\omega$, $0<\omega<0.5$, равен $2$, если сдвиг между синусоидами не равен $\pi/2$, и равен $1$ в случае комплексной экспоненты. Ранг вещественного синусоидального сигнала, равен  $2$ при тех же ограничениях на частоту. Ранги же комплексной и вещественной констант равны $1$ --- их можно рассматривать как частный случай с $\omega = 0$.

\section{Применение теории возмущений к оцениванию сигнала в SSA и CSSA}
\label{sec:cssa_theor}
% и \cite{Konstantinov}

Опишем коротко технику возмущений, используемую далее для получения ошибок оценки сигнала.

Наблюдаем комплексный временной ряд $\tX$ длины $N$, данный ряд представляется как $\tX = \tS + \tR$, где $\tS$~--- сигнал ранга $r$, $\tR$~--- возмущение.  Возьмем некоторую длину окна $L$, $L>r$.

В \cite{Nekrutkin} вводится разложение восстановления сигнала в модели $\tS(\delta) = \tS + \delta \tR$, что соответствует $\mathbf{H}(\delta) = \mathbf{H} + \delta \mathbf{E}$, где $\mathbf{H}(\delta) = \calT_L \tS(\delta)$, $\mathbf{H} = \tS(\delta)\tS$, $\delta\mathbf{E} = \delta \tR$, и рассматривается линейный по $\delta$ член ошибки восстановления, называемый первым порядком ошибки восстановления.

Рассмотрим возмущение ряда $\tR$ с $\delta = 1$, его траекторная матрица $\mathbf{E}$. Первый порядок ошибки восстановления обозначим как $\tF^{(1)} = \mathcal{H}(\mathbf{H}^{(1)})$.

На основе результатов из \cite[стр.12]{Konstantinov} и теоремы 2.1 из \cite{Nekrutkin} была получена следующая формула для $\mathbf{H}^{(1)}$ в случае достаточно маленького возмущения.
\begin{equation} \label{eq:main}
	\mathbf{H}^{(1)} = \mathbf{P}_0 \mathbf{E} \mathbf{Q}^{\perp}_0 + \mathbf{P}^{\perp}_0 \mathbf{E},
\end{equation}
где $\mathbf{P}^{\perp}_0$~--- проектор на пространство столбцов $\mathbf{H}$, $\mathbf{Q}^{\perp}_0$~--- проектор на пространство строк $\mathbf{H}$, $\mathbf{P}_0 = \mathbf{I} - \mathbf{P}^{\perp}_0$, $\mathbf{I}$~--- единичная матрица.

Теперь перейдём непосредственно к получению ошибок оценки сигнала и изложению результатов работы.

\subsection{Сравнение CSSA и SSA в случае совпадающих пространств сигналов}

Обозначим за:

$\tF^{(1)} = (f^{(1)}_1, \ldots, f^{(1)}_N)$ первый порядок ошибки восстановления $\tS$ с возмущением $\tR$ метода CSSA,

$\tF^{(1)}_{\Re} = (f^{(1)}_{\Re,1}, \ldots, f^{(1)}_{\Re, N})$ первый порядок ошибки восстановления $\Re(\tS)$ с возмущением $\Re(\tR)$ метода SSA,

$\tF^{(1)}_{\Im} = (f^{(1)}_{\Im,1}, \ldots, f^{(1)}_{\Im, N})$ первый порядок ошибки восстановления $\Im(\tS)$ с возмущением $\Im(\tR)$ метода SSA.


\begin{theorem}\label{th:sum}
Пусть пространства столбцов траекторных матриц рядов $\tS$, $\Re(\tS)$ и $\Im(\tS)$ совпадают и то же самое верно для пространств строк.
Тогда при любом достаточно малым возмущении $R$ $$\tF^{(1)} = \tF^{(1)}_{\Re} + \iu\tF^{(1)}_{\Im}.$$
\end{theorem}

Теорема непосредственно следует из линейности вхождения $\mathbf{E}$ в формулу \eqref{eq:main} и линейности диагонального усреднения.

Заметим, что хотя в утверждении теоремы возмущение $R$ может быть любым по форме, однако теорема имеет практическое применение только если первый порядок ошибки адекватно описывает полную ошибку.

\subsubsection{Случайное возмущение}

Рассмотрим случайное возмущение $\tR$.

Для дальнейших рассуждений приведём известный результат.
\begin{lemma} \label{std:disp}
Пусть $\zeta = \xi + \iu\eta$. Тогда $\mathbb{D}\zeta = \mathbb{D}\xi + \mathbb{D}\eta$.
\end{lemma}
%\begin{proof}
%\begin{multline*}
%	\mathbb{D}\zeta = \mathbb{E}(|\zeta - \mathbb{E}\zeta|^2) = \mathbb{E}(|(\xi - \mathbb{E}\xi) + \iu((\eta - \mathbb{E}\eta))|^2) = \\
%	= \mathbb{E}((\xi - \mathbb{E}\xi)^2 + (\eta - \mathbb{E}\eta)^2)= \mathbb{E}(\xi - \mathbb{E}\xi)^2 + \mathbb{E}(\eta - \mathbb{E}\eta)^2 = \mathbb{D}\xi + \mathbb{D}\eta.
%\end{multline*}
%\end{proof}

\begin{corollary}[из теоремы {\ref{th:sum}}] \label{st:dispsum}
	Пусть выполнены условия теоремы \ref{th:sum}.
	Тогда для любого $l$, $1\le l \le N$,
	\begin{equation} \label{eq:dispsum}
		\mathbb{D}f^{(1)}_l = \mathbb{D}f^{(1)}_{\Re, l} + \mathbb{D}f^{(1)}_{\Im, l}.	
	\end{equation}
\end{corollary}

Утверждение получается автоматически из теоремы \ref{th:sum} и леммы \ref{std:disp}.

\subsubsection{Возмущение в виде выброса}

Рассмотрим возмущение $\tR = (0, \ldots, 0, a + \iu b, 0, \ldots, 0)$, где $a + \iu b$ стоит на $k$-ой позиции.

\begin{proposition}\label{st:RMSEinv}
	Для рядов с сигналом $\tS$ и выбросами $a_1 + \iu b_1$, $a_2 + \iu b_2$ на позициях $k$, таких что $|a_1 + \iu b_1| = |a_2 + \iu b_2|$, модули первого порядка ошибки совпадают поэлементно.
\end{proposition}
\begin{proof}
	По формуле \eqref{eq:main}
	$$f^{(1)}_l = (a_1 + \iu b_1)c_l.$$
	Для ряда с сигналом $S$ и выбросом $a_1 + \iu b_2$ на позиции $k$ имеем
	$$|f^{(1)}_l| = |(a_1 + \iu a_2)c_l| = |a_1 + \iu a_2|\cdot|c_l|. $$
	Рассмотрим второй выброс $a_2 + \iu b_2$, равный первому по модулю.
	Для ряда с сигналом $\tS$ и выбросом $a_2 + \iu b_2$ на позиции $k$ получаем
	$$|f^{(1)}_l| = |(a_2 + \iu b_2)c_l| = |a_2 + \iu b_2|\cdot|c_l| = |a_1 + \iu b_1|\cdot|c_l|. $$
\end{proof}


\subsubsection{Случай двух зашумленных синусоид}
Пусть сигнал $\tS$ имеет вид
\begin{equation}
\label{eq:general_ts}
s_l = A\cos(2 \pi\omega l + \phi_1) + \iu B\cos(2 \pi\omega l + \phi_2),
\end{equation}
где $0<\omega\le 0.5$ и $0\le\phi_i < 2\pi$.
Заметим, что случай $|\psi_2-\psi_1| = \pi/2$ и $A=B$ соответствует комплексной экспоненте.

Пусть возмущение $\tR$~--- шум, т.е. случайный вектор с нулевым матожиданием и достаточно малой дисперсией.

\begin{corollary}[из теоремы {\ref{th:sum}}]
Для комплексного ряда вида \eqref{eq:general_ts}, кроме случая $|\psi_2-\psi_1| = \pi/2$ и $A=B$,  выполняется формула \eqref{eq:dispsum}.
\end{corollary}
Выполнение условий теоремы {\ref{th:sum}} (совпадение столбцовых и строковых траекторных пространств сигналов) для ряда вида \eqref{eq:general_ts} следует из результатов работы \cite{Golyandina.Stepanov2005}.

\begin{remark}
Численные эксперименты, проведённые в \cite{Golyandina.etal2013}, показывают, что для сигнала в виде комплексной экспоненты суммарная MSE CSSA-оценки сигнала равна полусумме суммарных MSE SSA-оценок сигнала его вещественной и мнимой частей.
\end{remark}

\begin{prop*}
	Для случая комплексной экспоненты, с возмущением $\tR$ выполняется
	\begin{equation} \label{eq:expdisp}
		\mathbb{D}(f^{(1)}_l) \stackrel{?}{=} \frac{1}{2}[\mathbb{D}(f^{(1)}_{\Re, l}) + \mathbb{D}(f^{(1)}_{\Im, l})].
	\end{equation}
\end{prop*}

Формула \eqref{eq:expdisp} была проверена числено. Приведём пример подобной численной проверки.
Сигнал $$s_l = e^{\iu 2 \pi l / 30},$$
с параметрами $\sigma^2 = 0.1$, $N = 5$, $L = 3$.

\begin{table}[H]
	\begin{center}
		\caption{Оценки дисперсий.}
		\label{tab:pi_div_2}
		\begin{tabular}{|c|c|c|c|c|c|}
			\hline
			$l$	& 1 & 2 & 3 & 4 & 5\\
			\hline
			$\hat{\mathbb{D}}(f^{(1)}_l)$ & 0.015  & 0.009  & 0.007 & 0.009 & 0.015\\
			\hline
			$\hat{\mathbb{D}}(f^{(1)}_{\Re, l})$ & 0.011 & 0.008 & 0.007 & 0.009 & 0.011\\
			\hline
			$\hat{\mathbb{D}}(f^{(1)}_{\Im, l})$ & 0.011  & 0.008  & 0.007 & 0.009 & 0.011\\
			\hline
		\end{tabular}
	\end{center}
\end{table}

Из таблицы \ref{tab:pi_div_2} видно выполнение предположения во всех точках, за исключением концов. Это можно объяснить медленной сходимостью по длине ряда на краях.

\subsubsection{Случай зашумлённой константы}

Рассматриваем ряд с $s_n = c_1 + ic_2$ и матрицу шума $\mathbf{E}$ с дисперсиями вещественной и мнимой частей $\sigma_1$ и $\sigma_2$.\\
Сингулярные векторы такого ряда являются нормированными векторами с одинаковыми компонентами, они также сингулярные для вещественной и мнимой части. Тогда выполняются условия теоремы \ref{th:sum} и

$$f^{(1)} = f^{(1)*}_{\Re(S)} + if^{(1)*}_{\Im(S)}.$$

В данном случае $\Re(S)$ и $\Im(S)$ являются вещественными константами.

В работе \cite{Vlas2008} была получена аналитическая формула для дисперсии каждого элемента вещественных констант, в нашем случае $f^{(1)}_{\Re}$ и $f^{(1)}_{\Im}$.

Обозначим $L = \alpha N$, $\alpha \leq \frac{1}{2}$, $\lambda = \lim_{N\to\infty} 2 l / N$, воспользовавшись утверждением \ref{st:dispsum}, получаем

$$
\mathbb{D} f^{(1)}_l = \mathbb{D} (f^{(1)}_{\Re, l}) + \mathbb{D} (f^{(1)}_{\Im, l}) \sim \frac{\sigma^2_1 + \sigma^2_2}{N}
\begin{cases}
	D_1(\alpha, \lambda), &\text{$0 \leq \lambda \leq 2 (1 - 2\alpha)$}\\
	D_2(\alpha, \lambda), &\text{$2 (1 - 2\alpha) < \lambda < 2\alpha$}\\
	D_3(\alpha, \lambda), &\text{$2\alpha \leq \lambda \leq 1$}
\end{cases},
$$
где
\begin{gather*}
	D_1(\alpha, \lambda) = \frac{1}{12 \alpha^2(1 - \alpha)^2} (\lambda^2(\alpha + 1) - 2\lambda\alpha(1 + \alpha)^2 + 4 \alpha(-3\alpha + 3 + 2\alpha^2))\\
	D_3(\alpha, \lambda) = \frac{1}{6 \alpha^2\lambda^2 (\alpha - 1)} (\lambda^4 + 2\lambda^3(3\alpha -2 -3\alpha^2) + \\
	+ 2\lambda^2(3 - 9\alpha + 12\alpha^2 - 4\alpha^3) + 4\lambda(4 \alpha^4 - 4\alpha^3 - 3\alpha^2 + 4\alpha - 1) +\\
	+ 8\alpha - 56 \alpha^2 + 144\alpha^3 - 160\alpha^4 + 64\alpha^5\\
	D_3(\alpha, \lambda) = \frac{2}{3\alpha}.\\
\end{gather*}

Формулы выписаны только до середины ряда из симметричности дисперсии первого порядка ошибки относительно середины ряда.
 

\subsection{Случай константных сигналов с выбросом}
Рассматриваем сигнал $\tS = (c_1 + \iu c_2, \ldots, c_1 + \iu c_2)$, возмущённый выбросом $a_1 + \iu a_2$ на позиции $k$, т.е. ряд $\tR$ состоит из нулей кроме значения $a_1 + \iu a_2$ на $k$-м месте. Исходя из теоремы \ref{th:sum}, достаточно уметь вычислять первый порядок ошибки восстановления сигнала $\tS = (c, \ldots, c)$, возмущённого выбросом $a$ на позиции $k$.

В работе \cite{NekrutkinPerp} была получен частный случай формулы \eqref{eq:main} для вещественных сигналов единичного ранга:
$$\mathbf{H}^{(1)} = -U^{\mathrm{T}} \mathbf{E} V U V^{\mathrm{T}} + U U^{\mathrm{T}} \mathbf{E} + \mathbf{E} V V^{\mathrm{T}},$$
где $U$, $V$~--- сингулярные вектора матрицы $\mathbf{H}$.

Подстановкой $U = \{1/\sqrt{L}\}^{L}_{i = 1},\, V = \{1/\sqrt{K}\}^{K}_{i = 1}$, $K = N - L + 1$ и последующим диагональным усреднением матрицы $\mathbf{H}^{(1)}$ был получен явный вид первого порядка ошибки восстановления.

Приведем результат для случая $k \leq \min(L/2, K - L)$ и $L < K$
$$f^{(1)}_l = \frac{a}{{LK}}
\begin{cases}
	(L + K - k), & \text{$1 \leq l \leq k$}\\
	\frac{1}{l}(L + K - l)k, & \text{$k < l \leq L$}\\
	\frac{1}{L}K(L + k - l), &\text{$L < l < L + k$}\\
	0, &\text{$L + k \leq l \leq K$}\\
	\frac{1}{N - l + 1}(K - l)(L - k), &\text{$K < l < K + k$}\\
	-k, &\text{$K + k \leq l \leq N $}
\end{cases}.$$

\begin{remark}
Из данной формулы видно, что при фиксированном $L$ первый порядок ошибки не стремится к $0$ с ростом $N$, тогда как численные эксперименты показывают, что полная ошибка восстановления стремится к $0$ с ростом $N$. Как показано в следующем разделе, это следствие того, что полная ошибка не описывается ее первым порядком. Если же $L$ и $K$ пропорциональны $N$, то первый порядок ошибки стремится к нулю.
\end{remark}

\section{Сравнение SSA и CSSA по трудоёмкости}
\label{sec:cssa_comp}

Вопрос времени работы метода играет немаловажную роль и особенно актуален при анализе рядов большой длины. Исходя из этого, осмысленным кажется сравнение трудоёмкостей реализаций CSSA и SSA, представленных в пакете \cite{Korobeynikov.etal2014}.

Ключевым различием реализаций выступает метод для вычисления SVD. А конкретнее способ вычисления быстрого умножения вектора на ганкелеву матрицу. Произведение вычисляется через DTF или дискретное преобразование Фурье, которое вычисляется через FFT или быстрое преобразование Фурье, данная идея изложена подробнее в \cite{Korobeynikov2010}. Отличие реализации SSA от реализации CSSA заключается в способе умножения через FFT, но в обоих случаях асимптотика составляет $\mathcal{O}(N \log N)$, где $N$~--- длина ряда.

Итого, асимптотическая трудоёмкость реализации SSA и CSSA совпадает и равна $\mathcal{O}(k N \log N + k^2 N)$, где $k$~--- число компонент для SVD, обоснование такой трудоёмксоти приведено в \cite{Korobeynikov2010}.

\section{Численное сравнение первого порядка ошибки и полной ошибки оценивания сигнала}
\label{sec:results}

В реальности нас интересует полная ошибка восстановления, однако, теоретические результаты в работе были получены для первого порядка. Приведём ряд численных экспериментов, целью которых было продемонстрировать, насколько точно первый порядок описывает полную ошибку восстановления.

Для случая зашумленных гармоник рассмотрен пример с сигналом $s_l = \cos(2 \pi l / 10) + \iu\cos(2 \pi l / 10 + \pi/4)$, $\sigma^2 = 0.1$, $N = 9$, $L = 5$. Результат для одной из реализаций шума представлен на рис.~\ref{fig:harm_noise}.

\begin{figure}[H]
	\begin{center}
		\includegraphics[width=0.6\linewidth]{img/first_vs_full_re.pdf}
		\caption{Вещественные части первого порядка и полной ошибок.}
		\label{fig:harm_noise}
	\end{center}
\end{figure}

Из графика видно, что ошибки практически совпадают даже при маленьких $L$ и $N$. Аналогичные численные эксперименты подтверждают, что для комплексной экспоненты также есть такое совпадение.

Для случая возмущения в виде выброса был рассмотрен пример с сигналом $s_l = 1 + \iu 1$, с возмущением в виде выброса $a_1 + \iu a_2 = 10 + \iu 10$ на позиции $k = L - 1$. Результаты представлены в таблице~\ref{tab:const_outl}.

\begin{table}[H]
	\begin{center}
		\caption{Максимальное различие первого порядка и полной ошибок.}
		\label{tab:const_outl}
		\begin{tabular}{|c|c|c|c|c|}
			\hline
			$N$	& 50 & 100 & 400 & 1600 \\
			\hline
			$L = N / 2$ & 0.1313  & 0.0419  & 0.0033 & 0.0002 \\
			\hline
			$L = 20$ & 0.3074  & 0.1965  & 0.5655 & 0.6720 \\
			\hline
		\end{tabular}
	\end{center}
\end{table}

Аналогичные численные эксперименты показывают, что при расположении выброса в середине ряда результаты качественно совпадают, при $L = N/2$ различие стремится к $0$, при $L = 20$ не стремится к $0$.

Численные результаты показывают, что для случая зашумленных гармоник первый порядок адекватно оценивает полную ошибку восстановления сигнала в каждой точке при любых рассматриваемых параметрах сигналов.

Однако для случая возмущения в виде выброса это верно, только когда $L$ и $K$ пропорциональны $N$.

Все численные результаты были получены при помощи пакета~\cite{Korobeynikov.etal2014}.

%\begin{figure}[!htb]
%    \centering
%    \begin{subfigure}{.45\textwidth}
%        \centering
%        \includegraphics[width=\textwidth]{img/noise}
%        \caption*{Зашумленный синус.}
%    \end{subfigure}
%    \begin{subfigure}{.45\textwidth}
%        \centering
%        \includegraphics[width=\textwidth]{img/outlier}
%        \caption*{Константа с выбросом.}
%    \end{subfigure}
%    \caption{Сравнение первого порядка ошибки и полной ошибкой.}
%    \label{fig:order1}
%\end{figure}

\section{Заключение}
\label{sec:conclusions}

В работе удалось подвести теоретическую базу под имеющиеся ранее численные результаты (\cite{Golyandina.etal2013}) по сравнению CSSA и SSA для двух зашумленных гармоник с одинаковой частотой и сдвигом, не равным $\pi/2$. Для зашумленной комплексной экспоненты был получен более общий, нежели имеющиеся ранее, численный результат.
Результаты показывают, что только в случае сигнала в виде комплексной экспоненты применение CSSA имеет смысл с точки зрения уменьшения ошибки восстановления сигнала. В случае константного сигнала была получена аналитическая формула для выражения дисперсии первого порядка ошибки восстановления.

Для константного ряда с выбросами был получен явный вид первого порядка ошибок оценки сигнала в каждой точке.

Для обоих случаев было численно исследовано соотношение между первым порядком ошибки и полной ошибкой.
В случае случайного возмущения оказалось, что первый порядок ошибки практически совпадает с полной ошибкой. Однако в случае неслучайного возмущения выбросом это не так и требуются дополнительные условия на пропорциональность длины окна $L$ длине ряда $N$.

В дальнейшем планируется теоретически подтвердить численный результат для случая комплексной экспоненты, подробнее изучить условие на соотношение сигнала и возмущения, обобщить полученные результаты для полной ошибки восстановления.



\bibliographystyle{ugost2008}
\bibliography{literature}

\end{document}
