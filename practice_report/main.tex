\documentclass[ucs, notheorems, handout]{beamer}

\usetheme[numbers,totalnumbers,minimal,nologo]{Statmod}
\usefonttheme[onlymath]{serif}
\setbeamertemplate{navigation symbols}{}
\setbeamercolor{alerted text}{fg=blue}
\usepackage{physics}

\mode<handout> {
    \usepackage{pgfpages}
    \setbeameroption{show notes}
    \pgfpagesuselayout{2 on 1}[a4paper, border shrink=5mm]
    \setbeamercolor{note page}{bg=white}
    \setbeamercolor{note title}{bg=gray!10}
    \setbeamercolor{note date}{fg=gray!10}
}

\usepackage[utf8x]{inputenc}
\usepackage[T2A]{fontenc}
\usepackage[english, russian]{babel}
\usepackage{tikz}
\usepackage{ragged2e}
\usepackage{graphicx}
\usepackage{subfigure}


\usepackage{latexsym,amssymb}
\usepackage{amsmath}
\usepackage{amsfonts}
\usepackage{amsthm}

\DeclareMathOperator{\rk}{rk}
\DeclareMathOperator{\med}{med}
\DeclareMathOperator{\diag}{diag}
\DeclareMathOperator*{\argmin}{argmin}
\newcommand{\tX}[1]{\mathsf{#1}}
\newcommand{\iu}{\mathrm{i}\mkern1mu}

\title[Ошибки CSSA]
{Ошибки оценивания сигнала с помощью комплексного анализа сингулярного спектра}

\subtitle
{}

\author[Сенов М.А.] 
{Сенов Михаил Андреевич}

\institute[Санкт-Петербургский Государственный Университет]{%
    \small
    Санкт-Петербургский государственный университет\\
    Прикладная математика и информатика\\
    Вычислительная стохастика и статистические модели\\
    \vspace{1.25cm}
    Отчет по преддипломной практике}

\date[Зачет]{Санкт-Петербург, 2022}

\begin{document}

\begin{frame}
  \titlepage
  \note{Научный руководитель  к.ф.-м.н., доцент Голяндина Н.Э.,\\
    кафедра статистического моделирования}
\end{frame}

\begin{frame}{Введение: Проблема}

$\tX{X} = (x_1, \ldots, x_{N})$ временной ряд длины $N$.\\
\vspace{1em}
\alert{Модель:} $\tX{X} = \tX{S} + \tX{R}$, $\tX{S}$ сигнал, $\tX{R}$ возмущение (шум или выброс).\\
\vspace{1em}
\alert{Задача:} Оценить сигнал $\tilde{\tX{S}} = F(\tX{X})$, $F$ --- используемый метод.\\
\vspace{1em}
\alert{Метод:} SSA (Singular Spectrum Analysis) [Golyandina et al., 2001] для вещественных рядов, CSSA (Complex Singular Spectrum Analysis) --- обобщение для комплексных рядов.\\
\vspace{1em}
\alert{Проблема:} Что лучше, с точки зрения величины ошибки $\tilde{\tX{S}} - \tX{S}$,

SSA$(\tX{X}_{\Re}) + \iu \text{SSA}(\tX{X}_{\Im})$ или CSSA$(\tX{X}_{\Re} + \iu \tX{X}_{\Im})$?

\note{Рассмотрим временной ряд длины $N$. В качестве модели возьмём представление временного ряда в виде суммы сигнала и возмущения. В данной работе, в качестве возмущения будет рассмотрен шум(случайная ошибка) и выброс(аномальный элемент). Задачей является оценка сигнала $\tX{S}$ по ряду $\tX{X}$. Данную задачу решает метод SSA в вещественном случае и его комплексное обобщение CSSA в комплексном. Имея вещественный и комплексный метод, возникает вопрос, а что лучше, с точки зрения величины ошибки оценки сигнала, применить вещественный метод отдельно к вещественной и отдельно к мнимой частям ряда или применить комплексный метод ко всему ряду? На данный вопрос мы постараемся ответить в работе.}
\end{frame}

\begin{frame}{Введение: Обозначения}
Рассмотрим временной ряд $\tX{X}=(x_1, \ldots, x_{N})$, $L$ длина окна, $K=N-L+1$.\\
\vspace{1em}
$\mathcal{M}_{r}$ --- пространство матриц размера $L \times K$ ранга не больше $r$.

$\mathcal{M}_{\mathcal{H}}$~--- пространство ганкелевых матриц $L\times K$.\\
\vspace{1em}
Оператор вложения $\mathcal{T}_L:\mathbb{R}^N \rightarrow \mathcal{M}_{\mathcal{H}}: \mathcal{T}_L (\tX{X}) = \mathbf{X} $,\\
$$\mathbf{X} = \begin{pmatrix}
           x_1 & x_2 & \ldots & x_{K}\\
           x_2 & x_3 & \ldots & x_{K+1}\\
           \vdots & \vdots & & \vdots\\
           x_{L} & x_{L+1} & \ldots & x_{N}
         \end{pmatrix}, K = N - L + 1,$$
$\mathbf{X}$~--- $L$-траекторная матрица $\tX{X}$.\\
\vspace{1em}
$\Pi_{r}:\mathcal{M}\rightarrow \mathcal{M}_r$,
$\Pi_{\mathcal{H}}:\mathcal{M} \rightarrow \mathcal{M}_{\mathcal{H}}$~--- проекторы на $\mathcal{M}_{r}$ и $\mathcal{M}_{\mathcal{H}}$ по норме Форбениуса.

    \note{Для дальнейшего обсуждения потребуется ввести ряд обозначений. Два матричных пространства, ранга не больше $r$ и ганкелевых матриц. Оператор вложения, переводящий ряд в ганкелеву матрицу, называемую траекторной матрицей ряда. Два матричных проектора по норме Фробениуса, на пространство ранга не больше $r$ и на пространство ганкелевых матриц.}

\end{frame}

\begin{frame}{Введение: Ошибка восстановления}
Временной ряд $\tX{X}=(x_1, \ldots, x_{N})$, $L$ --- длина окна, $r$ --- ранг оцениваемого сигнала (ранга траекторной матрицы сигнала).
    \begin{block}{Алгоритм SSA (CSSA) выделения сигнала}
\begin{equation*}
	\tilde{\tX{S}} = \mathcal{T}^{-1}_{L} \Pi_{\mathcal{H}} \Pi_{r} \mathcal{T}_L (\tX{X}).
\end{equation*}
\end{block}

\alert{Модель:} $\tX{X} = \tX{S}(\delta)$, где $\tX{S}(\delta) = \tX{S} + \delta \tX{R}$ длины $N$\\ $\mathbf{H} = \mathcal{T}_L(\tX{S})$.\\
\vspace{1em}
$\tilde{\tX{S}} = \mathcal{T}_L^{-1} \Pi_{\mathcal{H}} (\mathbf{H} + \delta\mathbf{H}^{(1)} + \delta^2\mathbf{H}^{(2)})$ из (Nekrutkin, 2008).\\
\vspace{1em}
$\tX{F} = \tilde{\tX{S}} - \tX{S} = \mathcal{T}_L^{-1} \Pi_{\mathcal{H}} (\delta\mathbf{H}^{(1)} + \delta^2\mathbf{H}^{(2)})$ ошибка восстановления.\\
\vspace{1em}
Рассматриваем $\delta = 1$,\\
\fbox{$\tX{F}^{(1)} = \mathcal{T}_L^{-1} \Pi_{\mathcal{H}}(\mathbf{H}^{(1)})$ первый порядок ошибки восстановления.}

\note{Во введенных выше обозначениях выразим чему равна оценка сигнала алгоритмом SSA(CSSA). Рассмотрим модель $\tX{S}(\delta)$. При такой модели знаем выражение оценки сигнала через константу по $\delta$, линейную компоненту и квадратичную по $\delta$, из работы \cite{NekrutkinPerp}. Отсюда естественным образом можно выразить ошибку восстановления через линейную и квадратичную компоненты по $\delta$. И взяв $\delta = 1$ обозначим за первый порядок ошибки диагонально усреднённую линейную по $\delta$ компоненту.}
\end{frame}

\begin{frame}{Ошибка восстановления: Формула для $\mathbf{H}^{(1)}$}

Итак, рассматриваем $\delta = 1$,
$\tX{X} = \tX{S}(1) = \tX{S} + \tX{R}$,

Тогда
$\tilde{\tX{S}} = \mathcal{T}_L^{-1} \Pi_{\mathcal{H}} (\mathbf{H} + \mathbf{H}^{(1)} + \mathbf{H}^{(2)}))$,

$\tX{F} = \tilde{\tX{S}} - \tX{S}$,

$\tX{F}^{(1)} = \mathcal{T}_L^{-1} \Pi_{\mathcal{H}}(\mathbf{H}^{(1)})$.\\
\vspace{1em}
На основе результатов из (Константинов,~2018) и (Некруткин,~2010) была получена формула для $\mathbf{H}^{(1)}$ в случае достаточно маленького возмущения:

\begin{equation*} \label{eq:main}
	\mathbf{H}^{(1)} = \mathbf{P}^{\perp}_0 \mathbf{E} \mathbf{Q}_0 + \mathbf{P}_0 \mathbf{E},
\end{equation*}
где $\mathbf{P}_0$~--- проектор на пространство столбцов $\mathbf{H}$, \\$\mathbf{Q}_0$~--- проектор на пространство строк $\mathbf{H}$,\\ $\mathbf{P}^{\perp}_0 = \mathbf{I} - \mathbf{P}_0$, $\mathbf{I}$~--- единичная матрица,\\
$\mathbf{E} = \mathcal{T}_L(\tX{R})$.

\note{Как вычисляется первый порядок ошибки? На основе результатов \cite{Nekrutkin} и \cite{Konstantinov} была получена следующая формула для линейной по $\delta$ компоненте. Соответственно, чтобы получить первый порядок ошибки восстановления, достаточно к полученной матрице применить диагональное усреднение.}
\end{frame}

\begin{frame}{Ошибка восстановления: Теорема}
    CSSA-$\tX{F}^{(1)}$ первый порядок ошибки восстановления $\tX{S}$ с возмущением $\tX{R}$ метода CSSA,

    SSA-$\tX{F}^{(1)}_{\Re}$ первый порядок ошибки восстановления $\Re(\tX{S})$ с возмущением $\Re(\tX{R})$ метода SSA,

    SSA-$\tX{F}^{(1)}_{\Im}$ первый порядок ошибки восстановления $\Im(\tX{S})$ с возмущением $\Im(\tX{R})$ метода SSA.

    \begin{block}{Теорема \label{th:sum}}
        Пусть пространства столбцов траекторных матриц рядов $\tX{S}$, $\Re(\tX{S})$ и $\Im(\tX{S})$ совпадают и то же самое верно для пространств строк.
    Тогда при любом достаточно малом возмущении $\tX{R}$
    $$\text{CSSA-}\tX{F}^{(1)} = \text{SSA-}\tX{F}^{(1)}_{\Re} + \iu\text{SSA-}\tX{F}^{(1)}_{\Im}.$$
    \end{block}
    Получается из линейности вхождения $\mathbf{E}$ в формулу
    $$\mathbf{H}^{(1)} = \mathbf{P}^{\perp}_0 \mathbf{E} \mathbf{Q}_0 + \mathbf{P}_0 \mathbf{E}.$$
    
\note{Центровым теоретическим результатом работы является теорема о выражении первого порядка ошибки CSSA через первый порядок ошибки для SSA. Она приведена на слайде и её суть заключается в ответе на вопрос, что лучше, применить SSA к вещественной и мнимой частям или CSSA ко всему ряду. Оказывается, что для рядов, у которых совпадают траекторные пространства, неважно, какой из этих двух подходов использовать. Теорема получается из линейности вхождения матрицы возмущения в формулу линейного по $\delta$ компонента ошибки.}
\end{frame}

\begin{frame}{Ошибка восстановления: Случайное возмущение}
$\text{CSSA-}\tX{F}^{(1)} = (\text{CSSA-}f^{(1)}_1, \ldots, \text{CSSA-}f^{(1)}_N)$,

$\text{SSA-}\tX{F}^{(1)}_{\Re} = (\text{SSA-}f^{(1)}_{\Re, 1}, \ldots, \text{SSA-}f^{(1)}_{\Re, N})$,

$\text{SSA-}\tX{F}^{(1)}_{\Im} = (\text{SSA-}f^{(1)}_{\Im, 1}, \ldots, \text{SSA-}f^{(1)}_{\Im, N})$\\
\vspace{1em}

Пусть возмущение $\tX{R}$~--- шум, т.е. комплексный случайный вектор с нулевым матожиданием.\\
\vspace{1em}
\begin{block}{Применение теоремы}
Пусть выполнены условия теоремы.
Тогда
\begin{equation*} \label{eq:dispsum}
\mathbb{D}(\text{CSSA-}f^{(1)}_l) = \mathbb{D}(\text{SSA-}f^{(1)}_{\Re, l}) + \mathbb{D}(\text{SSA-}f^{(1)}_{\Im, l}).	
\end{equation*}
\end{block}

\alert{Известно:} Пусть $\zeta = \xi + \iu\eta$. Тогда $\mathbb{D}(\zeta) = \mathbb{D}(\xi) + \mathbb{D}(\eta)$.

Такое свойство комплексных случайных величин является объяснением, почему в теореме не требуется независимость вещественной и мнимой частей шума.
\note{В случае случайного возмущения, теорема распространяется так же и на дисперсии, при чём для этого не требуется независимости вещественной и мнимой частей возмущения, поскольку утверждение про то, что дисперсия комплексной случайной величины равна сумме дисперсий её мнимой и вещественной частей, выполняется и без этого.}
\end{frame}

\begin{frame}{Шум: Две зашумлённые синусоиды}
\alert{Опр.:} $L$-рангом ряда называется ранг $L$-траекторной матрицы.\\
\vspace{1em}
Пусть сигнал $\tX{S}$ имеет вид
\begin{equation*}
\label{eq:general_ts}
s_l = A\cos(2 \pi\omega l + \phi_1) + \iu B\cos(2 \pi\omega l + \phi_2),
\end{equation*}
где $0<\omega\le 0.5$ и $0\le\phi_i < 2\pi$.

При $|\phi_2-\phi_1| = \pi/2$ и $A=B$,
$$s_l = Ae^{\pm \iu(2 \pi\omega l + \phi_1)}.$$
В данном случае $L$-ранг $s_l$ равен $1$, иначе $2$, $L$-ранг $\Re(s_l)$ и $\Im(s_l)$ равен $2$ (Степанов,~Голяндина,~2005).\\
\vspace{1em}
По-прежнему, возмущение $\tX{R}$~--- шум, т.е. случайный вектор с нулевым матожиданием и достаточно малой дисперсией.

\note{Посмотрим на конкретные сигналы и применения к ним теоремы. Для начала рассмотрим случай зашумлённых синусоид. При различие по фазе синусоид на $\pi/2$, данный пример вырождается в случай комплексной экспоненты. Причём, из работы \cite{Golyandina.Stepanov2005} известно, что для случая комплексной экспоненты ранг комплексного траекторного пространства равен 1, тогда как ранг вещественного и мнимого равен 2, а значит, для данного случае, теорема не выполняется. В качестве возмущения будем рассматривать шум.}
\end{frame}

\begin{frame}{Шум: MSE}
\begin{block}{Замечание}
    Для сигнала $s_l = A\cos(2 \pi\omega l + \phi_1) + \iu B\cos(2 \pi\omega l + \phi_2)$, не являющегося комплексной экспонентой, с возмущением $\tX{R}$ выполняется
    $$\mathbb{D}(\text{CSSA-}f^{(1)}_l) = \mathbb{D}(\text{SSA-}f^{(1)}_{\Re, l}) + \mathbb{D}(\text{SSA-}f^{(1)}_{\Im, l}).$$
\end{block}

Совпадение пространств для $s_l$ известно из (Степанов,~Голяндина,~2005).

\begin{block}{Замечание}
    Для сигнала $s_l = Ae^{\pm \iu(2 \pi\omega l + \phi_1)}$, с возмущением $\tX{R}$ выполняется
    $$\mathbb{D}(\text{CSSA-}f^{(1)}_l) \stackrel{?}{=} \frac{1}{2}[\mathbb{D}(\text{SSA-}f^{(1)}_{\Re, l}) + \mathbb{D}(\text{SSA-}f^{(1)}_{\Im, l})].$$
\end{block}
Показано эмпирически.

\note{В случае зашумлённой гармоники, не являющейся комплексной экспонентной, выполняется формула разложения дисперсии CSSA, то есть дисперсия CSSA равна сумме дисперсий SSA. Это верно из-за того, что выполняется теорема и траекторные пространства совпадают, что было показано в работе \cite{Golyandina.Stepanov2005}. В случае же комплексной экспоненты нами было выдвинуто предположение, что дисперсия CSSA равна полусумме дисперсий SSA. Данное предположение было проверено рядом численных экспериментов.}
\end{frame}

\begin{frame}{Выброс: Константный сигнал с выбросом}
Пусть сигнал $\tX{S}$ имеет вид
$$s_l = c_1 + \iu c_2.$$
Пусть возмущение $\tX{R} = (0, \ldots, a_1 + \iu a_2, \ldots, 0)$~--- выброс на позиции $k$.\\
\vspace{1em}
По теореме достаточно уметь выражать $\text{SSA-}\tX{F}^{(1)}_{\Re}$.\\
\vspace{1em}
Для сигнала $\Re(s_l)$ известна формула (Nekrutkin, 2008)
$$\mathbf{H}^{(1)} = -U^{\mathrm{T}} \mathbf{E} V U V^{\mathrm{T}} + U U^{\mathrm{T}} \mathbf{E} + \mathbf{E} V V^{\mathrm{T}},$$
где $U = \{1/\sqrt{L}\}^{L}_{i = 1},\, V = \{1/\sqrt{K}\}^{K}_{i = 1}$, $K = N - L + 1$.

\note{Рассмотрим случай возмущения выбросом, для простоты рассматриваем константный сигнал. Данный сигнал удовлетворяется теореме и, поэтому, для вычисления первого порядка ошибки для CSSA, нам достаточно знать первый порядок ошибки для SSA. Для этого случая известна формула для выражения $\mathbf{H}^{(1)}$, полученная в \cite{NekrutkinPerp}.}
\end{frame}

\begin{frame}{Выброс: Явный вид SSA-$f^{(1)}_{\Re,l}$}
$\text{SSA-}f^{(1)}_{\Re, l} = \big( \mathcal{T}_L^{-1} \Pi_{\mathcal{H}}(\mathbf{H}^{(1)})\big)_{l}$\\
\vspace{1em}
    Приведем результат для случая $k \leq \min(L/2, K - L)$ и $L < K$, где $K=N-L+1$:
$$\text{SSA-}f^{(1)}_{\Re, l} = \frac{a}{{LK}}
\begin{cases}
	(L + K - k), & \text{$1 \leq l \leq k$}\\
	\frac{1}{l}(L + K - l)k, & \text{$k < l \leq L$}\\
	\frac{1}{L}K(L + k - l), &\text{$L < l < L + k$}\\
	0, &\text{$L + k \leq l \leq K$}\\
	\frac{1}{N - l + 1}(K - l)(L - k), &\text{$K < l < K + k$}\\
	-k, &\text{$K + k \leq l \leq N $}
\end{cases}.$$

\alert{Замечание:} При фиксированном $L$ первый порядок ошибки не стремится к $0$ с ростом $N$.

\note{Подставив в эту формулу конкретные компоненты, и диагонально усреднив полученную матрицу, мы получили аналитический вид для первого порядка ошибки. На слайде приведена формула для выброса на определённой позиции, однако в общем случае она аналогична. Из формулы можно увидеть, что при фиксированном $L$ первый порядок ошибки не стремится к $0$ с ростом $N$. В дальнейшем мы увидим, что это следствие того, что в таком случае первый порядок плохо приближает полную ошибку, поскольку она стремится к $0$.}
\end{frame}

\begin{frame}{Выброс: График первого порядка для $k < L$}
Рассмотрим
$$s_l = 1 + \iu,$$
$\tX{R}$~--- выброс $10 + \iu 10$ на позиции $k = 1, 5, 10$, $N = 50$, $L = 20$.

% \begin{figure}[H]
%     \begin{center}
%         \includegraphics[, height = 2.2cm]{const_outl_graph_1.PNG}
%         \caption{График $\Re(\tX{X})$.}
%     \end{center}
% \end{figure}
\begin{figure}[H]
     \begin{center}
        \includegraphics[, height = 5cm]{const_outl_err_1.PNG}
        \caption{График $\Re(\text{CSSA-}\tX{F}^{(1)})$.}
    \end{center}
\end{figure}

\note{Приведём на графике, как выглядит первый порядок ошибки для выброса в начале ряда.}
\end{frame}

\begin{frame}{Выброс: График первого порядка для $L < k < K$}
Рассмотрим
$$s_l = 1 + \iu,$$
$\tX{R}$~--- выброс $10 + \iu 10$ на позиции $k = 26$, $N = 51$, $L = 10, 20, 25$.

% \begin{figure}[H]
%     \begin{center}
%         \includegraphics[, height = 2.2cm]{const_outl_graph_2.PNG}
%         \caption{График $\Re(\tX{X})$.}
%     \end{center}
% \end{figure}
\begin{figure}[H]
     \begin{center}
        \includegraphics[, height = 5cm]{const_outl_err_2.PNG}
        \caption{График $\Re(\text{CSSA-}\tX{F}^{(1)})$.}
    \end{center}
\end{figure}

\note{Приведём на графике, как выглядит первый порядок ошибки для выброса в середине ряда.}
\end{frame}

\begin{frame}{Сравнение первого порядка и полной ошибок: Зашумлённые гармоники}
Рассмотрим
$$s_l = \cos(2 \pi l / 10) + \iu\cos(2 \pi l / 10 + \pi/4),$$
$\tX{R}$~--- белый шум с $\sigma^2 = 0.1$, $N = 9$, $L = 5$.
\begin{figure}[H]
	\begin{center}
		\includegraphics[width=0.6\linewidth]{img/first_vs_full_re.pdf}
		\caption{Вещественные части первого порядка и полной ошибок.}
		\label{fig:harm_noise}
	\end{center}
\end{figure}
\alert{Вывод:} Первый порядок почти совпадает с полной ошибкой.

\note{Все прошлые результаты были получены нами для первого порядка ошибки. Однако, они не слишком осмыслены, если первый порядок не описывает полную ошибку. Для начала сравним первый порядок и полную ошибку для случая зашумленных гармоник. На приведённом графике видно, что первый порядок приближает первую ошибки даже при очень маленьком $N = 9$.}
\end{frame}

\begin{frame}{Сравнение первого порядка и полной ошибок: Константный сигнал с выбросом}

Рассмотрим
$$s_l = 1 + \iu,$$
$\tX{R}$~--- выброс $10 + \iu 10$ на позиции $k = L - 1$.

\begin{table}[H]
	\begin{center}
		\caption{Максимальное различие первого порядка и полной ошибок.}
		\label{tab:const_outl}
		\begin{tabular}{|c|c|c|c|c|}
			\hline
			$N$	& 50 & 100 & 400 & 1600 \\
			\hline
			$L = N / 2$ & 0.1313  & 0.0419  & 0.0033 & 0.0002 \\
			\hline
			$L = 20$ & 0.3074  & 0.1965  & 0.5655 & 0.6720 \\
			\hline
		\end{tabular}
	\end{center}
\end{table}

\alert{Вывод:} Первый порядок адекватно оценивает полную ошибку только в случае, когда $L$ и $K$ пропорциональны $N$.

\note{Сравним первый порядок с ошибкой восстановления для случая константного сигнала с выбросом. Из таблицы видно, что первый порядок адекватно описывает полную ошибку только в случае, когда $L$ и $K$ пропорциональны $N$. Что отсылает нас к тому, что при константном $L$ первый порядок не стремится к $0$ с ростом $N$.}
\end{frame}

\begin{frame}{Заключение}
    \begin{itemize}
        \item Теория: при совпадении траекторных пространств комплексного ряда первые порядки ошибок для CSSA и SSA совпадают.
        \item Численные эксперименты: при случайном возмущении первый порядок адекватно описывает полную ошибку.
        \item $s_l = A\cos(2 \pi\omega l + \phi_1) + \iu B\cos(2 \pi\omega l + \phi_2)$, сдвиг не $\pi / 2$
        $$\mathbb{D}(\text{CSSA-}f^{(1)}_l) = \mathbb{D}(\text{SSA-}f^{(1)}_{\Re, l}) + \mathbb{D}(\text{SSA-}f^{(1)}_{\Im, l}).$$
        $s_l = Ae^{\pm \iu(2 \pi\omega l + \phi_1)}$
        $$\mathbb{D}(\text{CSSA-}f^{(1)}_l) \stackrel{?}{=} \frac{1}{2}[\mathbb{D}(\text{SSA-}f^{(1)}_{\Re, l}) + \mathbb{D}(\text{SSA-}f^{(1)}_{\Im, l})].$$
        \item Численные эксперименты: в случае выброса первый порядок адекватно оценивает полную ошибку при $L = \alpha N$ для больших $N$. При малых $L$ это не так.
        \item Теория: для $s_l = c_1 + \iu c_2$ с выбросом была получена аналитическая формула первого порядка. При $L = \alpha N$ ошибка стремится к $0$.
    \end{itemize}
    
\note{В заключение можно отметить, что в работе были получены теоретические результаты для первого порядка ошибки. Так же были рассмотрены два основных случая возмущения, шума и выброса. Для случая шума был получен теор результат и было выдвинуто предположение, основанное на численных результатах. Для случая выброса была получена аналитическая формула. Так же, для этих случаев было показано, что первый порядок ошибки достаточно точно описывает полную ошибку восстановления.}
\end{frame}

\begin{frame}{Список литературы}
\begin{thebibliography}{10}
{\small
	\bibitem{Golyandina.etal2013}
	N.~Golyandina, A.~Korobeynikov, A.~Shlemov, and K.~Usevich (2015)
	\newblock Multivariate and {2D} extensions of singular spectrum analysis with
	the {R}ssa package.
	

	\bibitem{Golyandina.etal2001}
	N.~Golyandina, V.~Nekrutkin, and A.~Zhigljavsky (2001)
	\newblock {\em Analysis of Time Series Structure: {SSA} and Related
		Techniques}.

	
	\bibitem{Golyandina.Stepanov2005} 
	Д.~Степанов, Н.~Голяндина (2005)
	\newblock Варианты метода "Гусеница"{-SSA} для прогноза многомерных временных рядов.

	\bibitem{Nekrutkin}
	V.~Nekrutkin (2010)
	\newblock Perturbation expansions of signal subspaces for long signals.

	
	\bibitem{NekrutkinPerp}
	V.~Nekrutkin (2008)
	\newblock Perturbations in SSA.

	
	\bibitem{Konstantinov}
	А.~Константинов (2018)
	\newblock Некоторые задачи анализа временных рядов.
}	
\end{thebibliography}

\note{На данном слайде представлен список ключевых работ, использованных в работе.}
\end{frame}

\end{document}
